\section{db2-hash-routines}
\subsection{Building the library and registering the UDFs and SPs}
Login as the instance user. Change the {\tt DB2PATH} variable in the {\tt makertn} script for your environment.
\begin{verbatim}
DB2PATH=/home/db2inst1/sqllib
\end{verbatim}
Set {\tt DB2PATH} to the directory where DB2 is accessed. This is usually the instance home directory.\\
\\
After changing the above setting, start the script\\
\\
\begin{tabular}{@{} lll @{}}
Linux and AIX & & {\tt ./makertn}\\
Win32         & & {\tt makertn.bat}\\
\end{tabular}
\\\\
The UDFs and SPs are written in ANSI C and should compile on all platforms.
You can use the {\tt bldrtn} script in your {\tt sqllib/samples/c} directory as a good start.
The only thing that you have to do is to install APR and \mbox{APR-util}.
You can get APR and APR-util at \url{http://apr.apache.org/} \\
Furthermore you need to add the compiler and linker flags for APR (see makertn).\\
\\
To register the UDFs and SPs, connect to your database and run the script:
\begin{verbatim}
db2 -tvf register.ddl
\end{verbatim}
\newpage
\subsection{Description of the UDFs and SPs}
This library delivers five routines\footnote{see Appendix \ref{rtnreference} for a reference of the UDFs and SPs}:\\
\\
\hyperlink{hphpmd5}{\tt php\_md5}\\
\hyperlink{haprmd5}{\tt apr\_md5}\\
\hyperlink{haprcrypt}{\tt apr\_crypt}\\
\hyperlink{haprsha1}{\tt apr\_sha1}\\
\hyperlink{haprsha256}{\tt apr\_sha256}\\
\hyperlink{hvalidatepw}{\tt validate\_pw}\\
\\
The {\tt php\_md5} routine is compatible to the PHP md5 function.\\
The {\tt apr\_md5}, {\tt apr\_crypt} and {\tt apr\_sha1} routines are compatible to the Apache functions that are used in the {\tt htpasswd} utility.\\
The {\tt apr\_sha256} routine returns the identifier {\tt \{SHA256\}} plus the base64 encoded sha256 hash.\\
\\
{\tt validate\_pw} can be used to validate a password against a hash.
\paragraph{Note:}{In win32 environments {\tt apr\_crypt} returns the output of {\tt apr\_md5}.}