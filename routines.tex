\section{db2-hash-routines}
\subsection{Building the library and registering the UDFs and SPs}
Login as the instance user and run the script\\
\\
\begin{tabular}{@{} lll @{}}
Linux and AIX & & {\tt ./makertn}\\
Win32         & & {\tt makertn.bat}\\
\end{tabular}
\\\\
The {\tt makertn} script detects the DB2 instance directory and locates {\tt apr-1-config} and {\tt apu-1-config} automatically. If for some reason the script cannot set either one of the necessary variables, they have to be set manually.
Uncomment and change the following variables in the {\tt makertn} script.
\begin{verbatim}
DB2PATH=
APRPATH=
APUPATH=
\end{verbatim}
Set {\tt DB2PATH} to the directory where DB2 is accessed. This is usually the instance home directory.\\
Set APRPATH to where apr-1-config is located.\\
Set APUPATH to where apu-1-config is located.\\
\\
The UDFs and SPs are written in ANSI C and should compile on all platforms.\\
\\
The only requirements are APR and \mbox{APR-util}.
You can get APR and APR-util at \url{http://apr.apache.org/} \\
\\
To register the UDFs and SPs, connect to your database and run the script:
\begin{verbatim}
db2 -tvf register.ddl
\end{verbatim}
\newpage
\subsection{Description of the UDFs and SPs}
This library delivers the following routines\footnote{see Appendix \ref{rtnreference} for a reference of the UDFs and SPs}:\\
\\
\hyperlink{hbcrypt}{\tt bcrypt}\\
\hyperlink{hphpmd5}{\tt php\_md5}\\
\hyperlink{haprmd5}{\tt apr\_md5}\\
\hyperlink{haprcrypt}{\tt apr\_crypt}\\
\hyperlink{haprsha1}{\tt apr\_sha1}\\
\hyperlink{haprsha256}{\tt apr\_sha256}\\
\hyperlink{hvalidatepw}{\tt validate\_pw}\\
\\
The {\tt php\_md5} routine is compatible to the PHP md5 function.\\
The {\tt apr\_md5}, {\tt apr\_crypt}, {\tt apr\_sha1} and {\tt bcrypt} routines are compatible to the functions used in Apache's {\tt htpasswd} utility.\\
The {\tt apr\_sha256} routine returns the identifier {\tt \{SHA256\}} plus the base64 encoded sha256 hash.\\
\\
{\tt validate\_pw} can be used to validate a password against a hash.\\
\\
On systems with glibc2, the {\tt validate\_pw} routine will also validate hashes of the form \mbox{\tt \$id\$salt\$encrypted}. The following values of {\tt id} are supported:\\

\begin{tabular}{l|l}
 ID  &  Method                                                              \\ \hline
1    &  MD5                                                                 \\
2a   &  Blowfish (not in mainline glibc; added in some Linux distributions) \\
5    &  SHA-256 (since glibc 2.7)                                           \\
6    &  SHA-512 (since glibc 2.7)                                           \\
\end{tabular}

\paragraph{Note:}{In win32 environments {\tt apr\_crypt} returns the output of {\tt bcrypt}, if available. If {\tt bcrypt} is not available, the output of {\tt apr\_md5} is returned.}
