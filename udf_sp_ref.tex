\section{UDF and SP reference} \label{rtnreference}
\hypertarget{hbcrypt}{}
\subsection{bcrypt} \label{bcrypt1}
\begin{verbatim}
>>-BCRYPT--(--expression--)------------------------------------><

>>-BCRYPT--(--expression--,--hash--)---------------------------><
\end{verbatim}
bcrypt algorithm. The {\tt bcrypt} routine is compatible to the function used in Apache's {\tt htpasswd} utility.\\
\\
The argument can be a character string that is either a \mbox{CHAR} or \mbox{VARCHAR} not exceeding 4096 bytes.\\
\\
The result of the function is CHAR(60). The result can be null; if the argument is null, the result is the null value.\\
\\
Examples:
\begin{verbatim}
1)
    INSERT INTO USERS (username, password)
       VALUES ('test', bcrypt('testpwd'))

2)
    SELECT bcrypt( 'testpwd' ) FROM SYSIBM.SYSDUMMY1

    1
    ------------------------------------------------------------
    $2y$05$2jb66aPElSkNLT1t8e6dQepuCY2BP3JnYUh0xeV9r1PEoOGyOLkym

      1 record(s) selected.

3)
    CALL bcrypt('testpwd', ?)

      Value of output parameters
      --------------------------
      Parameter Name  : HASH
      Parameter Value : $2y$05$WYSu1X6PVA0Ra.aPSjrdv.S6hOp.AYSnNRT521rmLRjD4Mj9UY6ve

      Return Status = 0
\end{verbatim}
\newpage
\hypertarget{hsha256hex}{}
\subsection{sha256\_hex} \label{sha256hex}
\begin{verbatim}
>>-SHA256_HEX--(--expression--)--------------------------------><

>>-SHA256_HEX--(--expression--,--hash--)-----------------------><
\end{verbatim}
SHA256 algorithm. The {\tt sha256\_hex} routine returns a 64-character hexadecimal hash.\\
\\
The argument can be a character string that is either a \mbox{CHAR} or \mbox{VARCHAR} not exceeding 4096 bytes.\\
\\
The result of the function is CHAR(64). The result can be null; if the argument is null, the result is the null value.\\
\\
Examples:
\begin{verbatim}
1)
    INSERT INTO USERS (username, password)
       VALUES ('test', sha256_hex('testpwd'))

2)
    SELECT sha256_hex('testpwd') FROM SYSIBM.SYSDUMMY1

    1
    ----------------------------------------------------------------
    a85b6a20813c31a8b1b3f3618da796271c9aa293b3f809873053b21aec501087

      1 record(s) selected.

3)
    CALL sha256_hex('testpwd', ?)

      Value of output parameters
      --------------------------
      Parameter Name  : HASH
      Parameter Value : a85b6a20813c31a8b1b3f3618da796271c9aa293b3f809873053b21
    aec501087

      Return Status = 0
\end{verbatim}
\newpage
\hypertarget{hsha1hex}{}
\subsection{sha1\_hex} \label{sha1hex}
\begin{verbatim}
>>-SHA1_HEX--(--expression--)----------------------------------><

>>-SHA1_HEX--(--expression--,--hash--)-------------------------><
\end{verbatim}
SHA1 algorithm. The {\tt sha1\_hex} routine returns a 40-character hexadecimal hash.\\
\\
The argument can be a character string that is either a \mbox{CHAR} or \mbox{VARCHAR} not exceeding 4096 bytes.\\
\\
The result of the function is CHAR(40). The result can be null; if the argument is null, the result is the null value.\\
\\
Examples:
\begin{verbatim}
1)
    INSERT INTO USERS (username, password)
       VALUES ('test', sha1_hex('testpwd'))

2)
    SELECT sha1_hex('testpwd') FROM SYSIBM.SYSDUMMY1

    1
    ----------------------------------------
    98ef0758e6aac6f9a9e1197548c8190b72c9581d

      1 record(s) selected.

3)
    CALL sha1_hex('testpwd', ?)

      Value of output parameters
      --------------------------
      Parameter Name  : HASH
      Parameter Value : 98ef0758e6aac6f9a9e1197548c8190b72c9581d

      Return Status = 0
\end{verbatim}
\newpage
\hypertarget{hsha256}{}
\subsection{sha256} \label{sha256}
\begin{verbatim}
>>-SHA256--(--expression--+---------+--)-----------------------><
                          '-,--salt-'

>>-SHA256--(--expression--+---------+--,--hash--)--------------><
                          '-,--salt-'
\end{verbatim}
SHA256 algorithm. The {\tt sha256} routine returns a glibc2's crypt hash. If the system's crypt does not support sha-256, an \mbox{{\tt SQLSTATE 39702}} is returned.\\
\\
The argument can be a character string that is either a \mbox{CHAR} or \mbox{VARCHAR} not exceeding 4096 bytes.\\
An optional salt can be specified, which must be a eight-character string chosen from the set [a-z-Z0-9./]. If the salt is not exactly eight characters long, an \mbox{{\tt SQLSTATE 39703}} is returned. If the salt contains invalid characters, an \mbox{{\tt SQLSTATE 39704}} is returned.\\
\\
The result of the function is CHAR(55). The result can be null; if one of the arguments is null, the result is the null value.\\
\\
Examples:
\begin{verbatim}
1)
    INSERT INTO USERS (username, password)
       VALUES ('test', sha256('testpwd'))

2)
    SELECT sha256( 'testpwd' ) FROM SYSIBM.SYSDUMMY1

    1
    -------------------------------------------------------
    $5$S.LqPR7Z$273zPncMdmJ0dE1WdLldWVBmaHSDUDl8/tW8At8Hc0A

      1 record(s) selected.

3)
    CALL sha256('testpwd', ?)

      Value of output parameters
      --------------------------
      Parameter Name  : HASH
      Parameter Value : $5$vSDCZr2d$rfh.aDopE5l3lm26AwwcIYnuVdV7/9QBACWukqYyV3/

      Return Status = 0

4)
    SELECT sha256('testpwd', '12345678') FROM SYSIBM.SYSDUMMY1

    1
    -------------------------------------------------------
    $5$12345678$.oVAnOr/.FK8fYNiFPvoXPQvEOT9Calecygw6K9wIb9

      1 record(s) selected.

5)
    CALL sha256('testpwd', '12345678', ?)

    Value of output parameters
    --------------------------
    Parameter Name  : HASH
    Parameter Value : $5$12345678$.oVAnOr/.FK8fYNiFPvoXPQvEOT9Calecygw6K9wIb9

    Return Status = 0
\end{verbatim}
\newpage
\hypertarget{hsha512}{}
\subsection{sha512} \label{sha512}
\begin{verbatim}
>>-SHA512--(--expression--+---------+--)-----------------------><
                          '-,--salt-'

>>-SHA512--(--expression--+---------+--,--hash--)--------------><
                          '-,--salt-'
\end{verbatim}
SHA512 algorithm. The {\tt sha512} routine returns a glibc2's crypt hash. If the system's crypt does not support sha-512, an \mbox{{\tt SQLSTATE 39702}} is returned.\\
\\
The argument can be a character string that is either a \mbox{CHAR} or \mbox{VARCHAR} not exceeding 4096 bytes.\\
An optional salt can be specified, which must be a eight-character string chosen from the set [a-z-Z0-9./]. If the salt is not exactly eight characters long, an \mbox{{\tt SQLSTATE 39703}} is returned. If the salt contains invalid characters, an \mbox{{\tt SQLSTATE 39704}} is returned.\\
\\
The result of the function is CHAR(98). The result can be null; if one of the arguments is null, the result is the null value.\\
\\
Examples:
\begin{verbatim}
1)
    INSERT INTO USERS (username, password)
       VALUES ('test', sha512('testpwd'))

2)
    SELECT sha512( 'testpwd' ) FROM SYSIBM.SYSDUMMY1

    1
    ---------------------------------------------------------------------------
    -----------------------
    $6$cD33haq7$dl.RqEaLamlesTPVzSIQr4N1MY3BsVZ76VS8qNte0IOIWO2XorMg8U797KKOFGm
    X8dJhT3WuF6p17HmvvoQ6Q/

      1 record(s) selected.

3)
    CALL sha512('testpwd', ?)

      Value of output parameters
      --------------------------
      Parameter Name  : HASH
      Parameter Value : $6$1W.m9JN1$Dh.VPl7vy.igGaeDUdDWw6ZlD0xufwDWm0ukpOYknPt
    djxiSM2yzWBkzHffalb/2axNHPqEi9UUzXUbSm4LGa/

      Return Status = 0

4)
    SELECT sha512('testpwd', '12345678') FROM SYSIBM.SYSDUMMY1

    1
    ---------------------------------------------------------------------------
    -----------------------
    $6$12345678$tlHrypdWTz6FqubBpgL/ePlxr4lZuQ8OK1zfV6zWUmGJSz.5kGWwQGjg69Qm1Bm
    3.DvILruqA61o3EHsxSoko1

      1 record(s) selected.

5)
    CALL sha512('testpwd', '12345678', ?)

      Value of output parameters
      --------------------------
      Parameter Name  : HASH
      Parameter Value : $6$12345678$tlHrypdWTz6FqubBpgL/ePlxr4lZuQ8OK1zfV6zWUmGJS
    z.5kGWwQGjg69Qm1Bm3.DvILruqA61o3EHsxSoko1

    Return Status = 0
\end{verbatim}
\newpage
\hypertarget{hphpmd5}{}
\subsection{php\_md5} \label{phpmd5}
\begin{verbatim}
>>-PHP_MD5--(--expression--)-----------------------------------><

>>-PHP_MD5--(--expression--,--hash--)--------------------------><
\end{verbatim}
MD5 hash. The {\tt php\_md5} routine is compatible to the PHP md5 function.\\
\\
The argument can be a character string that is either a \mbox{CHAR} or \mbox{VARCHAR} not exceeding 4096 bytes.\\
\\
The result of the function is CHAR(32). The result can be null; if the argument is null, the result is the null value.\\
\\
Examples:
\begin{verbatim}
1)
    INSERT INTO USERS (username, password)
       VALUES ('test', php_md5('testpwd'))

2)
    SELECT php_md5( 'testpwd' ) FROM SYSIBM.SYSDUMMY1

    1
    --------------------------------
    342df5b036b2f28184536820af6d1caf

      1 record(s) selected.

3)
    CALL php_md5('testpwd', ?)

      Value of output parameters
      --------------------------
      Parameter Name  : HASH
      Parameter Value : 342df5b036b2f28184536820af6d1caf

      Return Status = 0
\end{verbatim}
\newpage
\hypertarget{haprmd5}{}
\subsection{apr\_md5} \label{aprmd5}
\begin{verbatim}
>>-APR_MD5--(--expression--)-----------------------------------><

>>-APR_MD5--(--expression--,--hash--)--------------------------><
\end{verbatim}
Seeded MD5 hash. The {\tt apr\_md5} routine is compatible to the function used in Apache's {\tt htpasswd} utility.\\
\\
The argument can be a character string that is either a \mbox{CHAR} or \mbox{VARCHAR} not exceeding 4096 bytes.\\
\\
The result of the function is CHAR(37). The result can be null; if the argument is null, the result is the null value.\\
\\
Examples:
\begin{verbatim}
1)
    INSERT INTO USERS (username, password)
       VALUES ('test', apr_md5('testpwd'))

2)
    SELECT apr_md5( 'testpwd' ) FROM SYSIBM.SYSDUMMY1

    1
    -------------------------------------
    $apr1$GfVmOTyJ$n7F1Vkwl/kX8MLgTJq1lp1

      1 record(s) selected.

3)
    CALL apr_md5('testpwd', ?)

      Value of output parameters
      --------------------------
      Parameter Name  : HASH
      Parameter Value : $apr1$GfVmOTyJ$n7F1Vkwl/kX8MLgTJq1lp1

      Return Status = 0
\end{verbatim}
\newpage
\hypertarget{haprcrypt}{}
\subsection{apr\_crypt} \label{aprcrypt}
\begin{verbatim}
>>-APR_CRYPT--(--expression--)---------------------------------><

>>-APR_CRYPT--(--expression--,--hash--)------------------------><
\end{verbatim}
Unix crypt. The {\tt apr\_crypt} routine is compatible to the function used in Apache's {\tt htpasswd} utility.\\
\\
The argument can be a character string that is either a \mbox{CHAR} or \mbox{VARCHAR} not exceeding 4096 bytes.\\
\\
The result of the function is CHAR(13). The result can be null; if the argument is null, the result is the null value.\\
\\
Examples:
\begin{verbatim}
1)
    INSERT INTO USERS (username, password)
       VALUES ('test', apr_crypt('testpwd'))

2)
    SELECT apr_crypt( 'testpwd' ) FROM SYSIBM.SYSDUMMY1

    1
    -------------
    cqs7uOvz8KBlk

      1 record(s) selected.

3)
    CALL apr_crypt('testpwd', ?)

      Value of output parameters
      --------------------------
      Parameter Name  : HASH
      Parameter Value : cqs7uOvz8KBlk

      Return Status = 0
\end{verbatim}
\newpage
\hypertarget{haprsha1}{}
\subsection{apr\_sha1} \label{aprsha1}
\begin{verbatim}
>>-APR_SHA1--(--expression--)----------------------------------><

>>-APR_SHA1--(--expression--,--hash--)-------------------------><
\end{verbatim}
SHA1 algorithm. The {\tt apr\_sha1} routine is compatible to the function used in Apache's {\tt htpasswd} utility.\\
\\
The argument can be a character string that is either a \mbox{CHAR} or \mbox{VARCHAR} not exceeding 4096 bytes.\\
\\
The result of the function is CHAR(33). The result can be null; if the argument is null, the result is the null value.\\
\\
Examples:
\begin{verbatim}
1)
    INSERT INTO USERS (username, password)
       VALUES ('test', apr_sha1('testpwd'))

2)
    SELECT apr_sha1( 'testpwd' ) FROM SYSIBM.SYSDUMMY1

    1
    ---------------------------------
    {SHA}mO8HWOaqxvmp4Rl1SMgZC3LJWB0=

      1 record(s) selected.

3)
    CALL apr_sha1('testpwd', ?)

      Value of output parameters
      --------------------------
      Parameter Name  : HASH
      Parameter Value : {SHA}mO8HWOaqxvmp4Rl1SMgZC3LJWB0=

      Return Status = 0
\end{verbatim}
\newpage
\hypertarget{haprsha256}{}
\subsection{apr\_sha256} \label{aprsha256}
\begin{verbatim}
>>-APR_SHA256--(--expression--)--------------------------------><

>>-APR_SHA256--(--expression--,--hash--)-----------------------><
\end{verbatim}
SHA256 algorithm. The {\tt apr\_sha256} routine returns the identifier {\tt \{SHA256\}} plus the base64 encoded sha256 hash.\\
\\
The argument can be a character string that is either a \mbox{CHAR} or \mbox{VARCHAR} not exceeding 4096 bytes.\\
\\
The result of the function is CHAR(52). The result can be null; if the argument is null, the result is the null value.\\
\\
Examples:
\begin{verbatim}
1)
    INSERT INTO USERS (username, password)
       VALUES ('test', apr_sha256('testpwd'))

2)
    SELECT apr_sha256( 'testpwd' ) FROM SYSIBM.SYSDUMMY1

    1
    ----------------------------------------------------
    {SHA256}qFtqIIE8Maixs/NhjaeWJxyaopOz+AmHMFOyGuxQEIc=

      1 record(s) selected.

3)
    CALL apr_sha256('testpwd', ?)

      Value of output parameters
      --------------------------
      Parameter Name  : HASH
      Parameter Value : {SHA256}qFtqIIE8Maixs/NhjaeWJxyaopOz+AmHMFOyGuxQEIc=

      Return Status = 0
\end{verbatim}
\newpage
\hypertarget{hvalidatepw}{}
\subsection{validate\_pw} \label{validatepw}
\begin{verbatim}
>>-VALIDATE_PW--(--password--,--hash--)------------------------><

>>-VALIDATE_PW--(--password--,--hash--,--is_valid--)-----------><
\end{verbatim}
This routine can be used to validate a password against a hash.\\
\\
The two input arguments can be character strings that are either a CHAR or VARCHAR not exceeding 4096 bytes (password) and 120 bytes (hash). The second parameter (hash) must not be empty, otherwise an \mbox{{\tt SQLSTATE 39701}} is returned.\\
\\
The result of the routine is an INTEGER. If the password is valid, 1 is returned. If the password is not valid, 0 is returned. The result can be null; if the argument is null, the result is the null value.\\
\\
Examples:
\begin{verbatim}
1)
    SELECT validate_pw('testpwd', 'cqs7uOvz8KBlk') FROM SYSIBM.SYSDUMMY1"

    1
    -----------
              1

      1 record(s) selected.

2)
    CALL validate_pw('testpwd', 'cqs7uOvz8KBlk', ?)

      Value of output parameters
      --------------------------
      Parameter Name  : IS_VALID
      Parameter Value : 1

      Return Status = 0

3)
    CALL validate_pw('testpwd', '0123456789abcdef', ?)

      Value of output parameters
      --------------------------
      Parameter Name  : IS_VALID
      Parameter Value : 0

      Return Status = 0
\end{verbatim}
