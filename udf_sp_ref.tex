\section{UDF and SP reference} \label{rtnreference}
\hypertarget{hmd5}{}
\subsection{md5} \label{md5}
\begin{verbatim}
>>-MD5--(--expression--)---------------------------------------><

>>-MD5--(--expression--,--hash--)------------------------------><
\end{verbatim}
MD5 hash. The {\tt md5} routine is compatible to the PHP md5 function.\\
\\
The argument can be a character string that is either a \mbox{CHAR} or \mbox{VARCHAR} not exceeding 120 bytes.\\
\\
The result of the function is CHAR(32). The result can be null; if the argument is null, the result is the null value.\\
\\
Examples:
\begin{verbatim}
1)
    INSERT INTO USERS (username, password)
       VALUES ('test', md5('testpwd'))

2)
    SELECT md5( 'testpwd' ) FROM SYSIBM.SYSDUMMY1

    1
    --------------------------------
    342df5b036b2f28184536820af6d1caf

      1 record(s) selected.

3)
    CALL md5('testpwd', ?)

      Value of output parameters
      --------------------------
      Parameter Name  : HASH
      Parameter Value : 342df5b036b2f28184536820af6d1caf

      Return Status = 0
\end{verbatim}
\newpage
\hypertarget{haprmd5}{}
\subsection{apr\_md5} \label{aprmd5}
\begin{verbatim}
>>-APR_MD5--(--expression--)-----------------------------------><

>>-APR_MD5--(--expression--,--hash--)--------------------------><
\end{verbatim}
Seeded MD5 hash. The {\tt apr\_md5} routine is compatible to the Apache function that is used in the {\tt htpasswd} utility.\\
\\
The argument can be a character string that is either a \mbox{CHAR} or \mbox{VARCHAR} not exceeding 120 bytes.\\
\\
The result of the function is CHAR(37). The result can be null; if the argument is null, the result is the null value.\\
\\
Examples:
\begin{verbatim}
1)
    INSERT INTO USERS (username, password)
       VALUES ('test', apr_md5('testpwd'))

2)
    SELECT apr_md5( 'testpwd' ) FROM SYSIBM.SYSDUMMY1

    1
    -------------------------------------
    $apr1$HsTNH...$bmlPUSoPOF/Qhznl.sAq6/

      1 record(s) selected.

3)
    CALL apr_md5('testpwd', ?)

      Value of output parameters
      --------------------------
      Parameter Name  : HASH
      Parameter Value : $apr1$HsTNH...$bmlPUSoPOF/Qhznl.sAq6/

      Return Status = 0
\end{verbatim}
\newpage
\hypertarget{haprcrypt}{}
\subsection{apr\_crypt} \label{aprcrypt}
\begin{verbatim}
>>-APR_CRYPT--(--expression--)---------------------------------><

>>-APR_CRYPT--(--expression--,--hash--)------------------------><
\end{verbatim}
Unix crypt. The {\tt apr\_crypt} routine is compatible to the Apache function that is used in the {\tt htpasswd} utility.\\
\\
The argument can be a character string that is either a \mbox{CHAR} or \mbox{VARCHAR} not exceeding 120 bytes.\\
\\
The result of the function is CHAR(13). The result can be null; if the argument is null, the result is the null value.\\
\\
Examples:
\begin{verbatim}
1)
    INSERT INTO USERS (username, password)
       VALUES ('test', apr_crypt('testpwd'))

2)
    SELECT apr_crypt( 'testpwd' ) FROM SYSIBM.SYSDUMMY1

    1
    -------------
    cqs7uOvz8KBlk

      1 record(s) selected.

3)
    CALL apr_crypt('testpwd', ?)

      Value of output parameters
      --------------------------
      Parameter Name  : HASH
      Parameter Value : cqs7uOvz8KBlk

      Return Status = 0
\end{verbatim}
\newpage
\hypertarget{haprsha1}{}
\subsection{apr\_sha1} \label{aprsha1}
\begin{verbatim}
>>-APR_SHA1--(--expression--)----------------------------------><

>>-APR_SHA1--(--expression--,--hash--)-------------------------><
\end{verbatim}
SHA1 algorithm. The {\tt apr\_sha1} routine is compatible to the Apache function that is used in the {\tt htpasswd} utility.\\
\\
The argument can be a character string that is either a \mbox{CHAR} or \mbox{VARCHAR} not exceeding 120 bytes.\\
\\
The result of the function is CHAR(33). The result can be null; if the argument is null, the result is the null value.\\
\\
Examples:
\begin{verbatim}
1)
    INSERT INTO USERS (username, password)
       VALUES ('test', apr_sha1('testpwd'))

2)
    SELECT apr_sha1( 'testpwd' ) FROM SYSIBM.SYSDUMMY1

    1
    ---------------------------------
    {SHA}mO8HWOaqxvmp4Rl1SMgZC3LJWB0=

      1 record(s) selected.

3)
    CALL apr_sha1('testpwd', ?)

      Value of output parameters
      --------------------------
      Parameter Name  : HASH
      Parameter Value : {SHA}mO8HWOaqxvmp4Rl1SMgZC3LJWB0=

      Return Status = 0
\end{verbatim}
\hypertarget{hvalidatepw}{}
\subsection{validate\_pw} \label{validatepw}
\begin{verbatim}
>>-VALIDATE_PW--(--password--,--hash--)------------------------><

>>-VALIDATE_PW--(--password--,--hash--,--is_valid--)-----------><
\end{verbatim}
This routine can be used to validate a password against a hash.\\
\\
The two input arguments can be character strings that are either a CHAR or VARCHAR not exceeding 120 bytes. The second parameter (hash) must not be 
empty, otherwise an \mbox{{\tt SQLSTATE 39701}} is returned.\\
\\
The result of the routine is an INTEGER. If the password is valid, 1 is returned. If the password is not valid, 0 is returned. The result can be null; if the argument is null, the result is the null value.\\
\\
Examples:
\begin{verbatim}
1)
    SELECT validate_pw('testpwd', 'cqs7uOvz8KBlk') FROM SYSIBM.SYSDUMMY1"

    1          
    -----------
              1

      1 record(s) selected.

2)
    CALL validate_pw('testpwd', 'cqs7uOvz8KBlk', ?)

      Value of output parameters
      --------------------------
      Parameter Name  : IS_VALID
      Parameter Value : 1

      Return Status = 0

3)
    CALL validate_pw('testpwd', '0123456789abcdef', ?)

      Value of output parameters
      --------------------------
      Parameter Name  : IS_VALID
      Parameter Value : 0

      Return Status = 0
\end{verbatim}