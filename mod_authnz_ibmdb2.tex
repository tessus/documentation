% $Id$
%\documentclass[11pt,a4paper]{scrartcl}
%\documentclass[11pt,a4paper]{article}
\documentclass[11pt,letterpaper]{article}
\usepackage{tabularx}
\usepackage{array}
\usepackage{verbatim}
\newenvironment{sverbatim}{\begingroup\small\verbatim}{\endverbatim\endgroup}
\newcommand{\stt}[1]{\tt \small #1}
\usepackage{lastpage}
\usepackage{fancyhdr}
\usepackage[
  bookmarks=true, 
  bookmarksnumbered=true, 
  pdftitle={Documentation for mod auth(nz) ibmdb2}, 
  pdfauthor={Helmut K. C. Tessarek}, 
  pdfcreator={$ $Id$ $}, 
  colorlinks=false, 
  linkcolor={blue}, 
  citecolor={blue}, 
  backref=true, 
  breaklinks=true, 
  pdftex
]{hyperref}
\date{\today}
\author{Helmut K. C. Tessarek}
\title{Documentation for\\
mod\_auth(nz)\_ibmdb2\\
db2-hash-routines\\
scripts}
\begin{document}
\maketitle
\setcounter{page}{0}
\thispagestyle{empty}
\begin{abstract}
mod\_auth(nz)\_ibmdb2 is an Apache authentication module using IBM\textsuperscript{\textregistered}{} DB2\textsuperscript{\textregistered}{} as the backend database for storing user and group information. The module supports several encryption methods.
\end{abstract}

\vfill

\begin{center}
\href{http://mod-auth-ibmdb2.sourceforge.net}{http://mod-auth-ibmdb2.sourceforge.net}\\

{\tiny \rm $ $RCSfile$ $ }
{\tiny \rm $ $Date$ $ }
{\tiny \rm $ $Revision$ $ }

\end{center}
\newpage
\renewcommand{\thepage}{\roman{page}}
\tableofcontents
\newpage
\pagestyle{fancy}
\lhead{}
\chead{Documentation for mod\_auth(nz)\_ibmdb2}
\rhead{}
\rfoot{{\tiny \rm $ $Revision$ $}}
\cfoot{\thepage\ / \pageref*{LastPage}}
\lfoot{{\tiny \rm $ $Date$ $}}
\renewcommand{\headrulewidth}{0.4pt}
\renewcommand{\footrulewidth}{0.4pt}
\renewcommand{\thepage}{\arabic{page}}
\setcounter{page}{1}
\section{mod\_auth(nz)\_ibmdb2}

\subsection{Building mod\_auth(nz)\_ibmdb2}
Log in as root user.\\
\\
Change the {\tt DB2PATH} and the {\tt APXS} variables in the {\tt makemod} script according to your environment:
\begin{verbatim}
# Path settings

DB2PATH=/home/db2inst1/sqllib
APXS=/usr/local/apache/bin/apxs
\end{verbatim}
Set {\tt DB2PATH} to the directory where DB2 is accessed. This is usually the instance home directory.\\
\\
Set {\tt APXS} to the path that points to your apxs binary. The apxs binary is usually installed in the /$<$your apache home$>$/bin directory.\\
\\
After changing the above settings, run the script {\tt ./makemod}\\
\\
Last but not least the DB2 environment has to be set in the Apache startscript. This is done by sourcing the {\tt db2profile} script, which is located in {\tt DB2PATH}.\\

\subsection{Details on building mod\_auth(nz)\_ibmdb2}
To build the module the Apache utility {\tt apxs} is used. The {\tt EXTRA\_LFLAG} needs to be specified so that the module will find the db2 library during runtime.
\begin{verbatim}
DB2PATH=/home/db2inst1/sqllib

EXTRA_LFLAG="-Wl,-G,-blibpath:$DB2PATH/lib"     (for AIX)
EXTRA_LFLAG="-Wl,-rpath,$DB2PATH/lib"           (for Linux)
\end{verbatim}
\underline{mod\_authnz\_ibmdb2:}
\begin{verbatim}
apxs -c -ldb2 $EXTRA_LFLAG mod_authnz_ibmdb2.c
\end{verbatim}
\underline{mod\_auth\_ibmdb2:}
\begin{verbatim}
apxs -c -ldb2 $EXTRA_LFLAG mod_auth_ibmdb2.c
\end{verbatim}
\underline{mod\_auth\_ibmdb2 (Apache 1.x):}
\begin{verbatim}
apxs -c -ldb2 -lcrypt -lgdbm $EXTRA_LFLAG mod_auth_ibmdb2.c
\end{verbatim}
If the {\tt sqlcli1.h} header file cannot be found, add the {\tt -I} option to specify the
directory where {\tt sqlcli1.h} can be found.
If the db2 library cannot be found, add the {\tt -L} option to specify the
directory where {\tt libdb2.so} can be found.\\
\\
For example:
\begin{verbatim}
apxs -c -L/home/db2inst1/sqllib/lib        \
  -I/home/db2inst1/sqllib/include -ldb2    \
  $EXTRA_LFLAG mod_authnz_ibmdb2.c
\end{verbatim}
After creating the module, it has to be moved to the Apache module directory. This is also done with the apxs utility:\\
\\
\underline{mod\_authnz\_ibmdb2:}
\begin{verbatim}
apxs -i mod_authnz_ibmdb2.la
\end{verbatim}
\underline{mod\_auth\_ibmdb2:}
\begin{verbatim}
apxs -i mod_auth_ibmdb2.la
\end{verbatim}
\underline{mod\_auth\_ibmdb2 (Apache 1.x):}
\begin{verbatim}
apxs -i mod_auth_ibmdb2.so
\end{verbatim}
As the next step the DB2 environment has to be set in the Apache startscript. This is done by sourcing the {\tt db2profile} script, which is located in {\tt DB2PATH}.\\
\\
Finally, add the following directive to your httpd.conf and restart Apache:\\
\\
\underline{mod\_authnz\_ibmdb2:}
\begin{verbatim}
LoadModule authnz_ibmdb2_module modules/mod_authnz_ibmdb2.so
\end{verbatim}
\underline{mod\_auth\_ibmdb2:}
\begin{verbatim}
LoadModule ibmdb2_auth_module   modules/mod_auth_ibmdb2.so
\end{verbatim}
\underline{mod\_auth\_ibmdb2 (Apache 1.x):}
\begin{verbatim}
LoadModule ibmdb2_auth_module   modules/mod_auth_ibmdb2.so
AddModule  mod_auth_ibmdb2.c
\end{verbatim}

\subsection{Building a static module (Apache 1.3.x)}
Extract the Apache source code to a directory (e.g. {\tt /ext}). Change into the {\tt modules} directory of the Apache source tree:
\begin{verbatim}
cd /ext/apache_X.Y.ZZ/src/modules
\end{verbatim}
Extract mod\_auth\_ibmdb2-x.y.z.tar.gz into the {\tt modules} directory. Change into the {\tt mod\_auth\_ibmdb2} directory and copy the two files from the {\tt static} directory into the current one:
\begin{verbatim}
cd mod_auth_ibmdb2
cp static/.indent.pro .
cp static/Makefile.tmpl .
\end{verbatim}
Change into the {\tt src} directory:
\begin{verbatim}
cd ../..
\end{verbatim}
Add the following line to the {\tt Configuration} file and to the {\tt Configuration.tmpl} file:
\begin{verbatim}
AddModule modules/mod_auth_ibmdb2/mod_auth_ibmdb2.o
\end{verbatim}
Source the db2profile:
\begin{verbatim}
. /home/db2inst1/sqllib/db2profile
\end{verbatim}
Make the server by running {\tt make} in the {\tt src} directory. If you run {\tt httpd -l} after the executable has been built, you shoud see {\tt mod\_auth\_ibmdb2.c} in the list of compiled in modules.

\subsection{Installing the Manpages}
There is a {\tt man} directory in the path, where you have extracted the mod\_auth\_ibmdb2 package.\\ 
\\
Change into the man directory and run the script {\tt ./maninstall} 
\newpage

\subsection{Description of the module}
\emph{mod\_auth\_ibmdb2} is an Apache authentication module using IBM DB2 as the backend database for storing user and group information. The module is designed for Apache 2.0.x and 1.x and supports several encryption methods.\\
\emph{mod\_authnz\_ibmdb2} is designed for Apache 2.2.x and later and is based on the new authentication/authorization framework.\\
\\
Here is a list of the new directives\footnote{see Appendix \ref{default}} that come with the module:\\
\\
\small
\begin{tabular}{@{} ll >{\raggedright\arraybackslash}p{ 50ex } @{}}
{\tt AuthIBMDB2Database} & & database name \\

{\tt AuthIBMDB2Hostname} & & database server hostname for uncataloged databases \\

{\tt AuthIBMDB2Portnumber} & & database instance port \mbox{(default: 50000)} \\

{\tt AuthIBMDB2User} & & user for connecting to the DB2 database \\

{\tt AuthIBMDB2Password} & & password for connecting to the DB2 database \\

{\tt AuthIBMDB2UserTable} & & name of the user table \\

{\tt AuthIBMDB2GroupTable} & & name of the group table \\

{\tt AuthIBMDB2NameField} & & name of the user column in the table \mbox{(default: {\tt username})} \\

{\tt AuthIBMDB2GroupField} & & name of the group column in the table \mbox{(default: {\tt groupname})} \\

{\tt AuthIBMDB2PasswordField} & & name of the password column in the table \mbox{(default: {\tt password})} \\

{\tt AuthIBMDB2CryptedPasswords} & & passwords are stored encrypted \mbox{(default: {\tt yes})} \\

{\tt AuthIBMDB2KeepAlive} & & connection kept open across requests \mbox{(default: {\tt yes})} \\

{\tt AuthIBMDB2Authoritative} & & lookup is authoritative \mbox{(default: {\tt yes})} \\ 

{\tt AuthIBMDB2NoPasswd} & & just check, if user is in usertable \mbox{(default: {\tt no})} \\

{\tt AuthIBMDB2UserCondition} & & restrict result set \\

{\tt AuthIBMDB2GroupCondition} & & restrict result set \\

{\tt AuthIBMDB2UserProc} & & stored procedure\footnotemark[2] for user authentication \\

{\tt AuthIBMDB2GroupProc} & & stored procedure\footnotemark[2] for group authentication \\

{\tt AuthIBMDB2Caching} & & user credentials are cached \mbox{(default: {\tt off})} \\

{\tt AuthIBMDB2GroupCaching} & & group information is cached \mbox{(default: {\tt off})} \\

{\tt AuthIBMDB2CacheFile} & & path to cache file \mbox{(default: {\tt /tmp/auth\_cred\_cache})} \\

{\tt AuthIBMDB2CacheLifetime} & & cache lifetime in seconds \mbox{(default: {\tt 300})} \\
\end{tabular}
\footnotetext[2]{SP support is not available in mod\_auth\_ibmdb2 for Apache 1.x, see Appendix \ref{sp}}
\addtocounter{footnote}{1}
\normalsize

\newpage
\noindent
If {\tt AuthIBMDB2Authoritative} is {\tt Off}, then iff the user is not found in the database, let other authentication modules try to find the user. Default is {\tt On}. \\
\\
If {\tt AuthIBMDB2KeepAlive} is {\tt On}, then the server instance will keep the IBM DB2 server connection open.  In this case, the first time the connection is made, it will use the current set of Host, User, and Password settings.  Subsequent changes to these will not affect this server, so they should all be the same in every htaccess file. 
If you need to access multiple IBM DB2 servers for this authorization scheme from the same web server, then keep this setting {\tt Off} -- this will open a new connection to the server every time it needs one.  The values of the database and various tables and fields are always used from the current {\tt .htaccess} file settings.\\
\\
If {\tt AuthIBMDB2NoPasswd} is {\tt On}, then any password the user enters will be accepted as long as the user exists in the database.\\ Setting this also overrides the setting for {\tt AuthIBMDB2PasswordField} to be the same as {\tt AuthIBMDB2NameField} (so that the SQL statements still work when there is no password at all in the database, and to remain backward-compatible with the default values for these fields.)\\
\\
For groups, we use the same {\tt AuthIBMDB2NameField} as above for the user ID, and {\tt AuthIBMDB2GroupField} to specify the group name.\\ {\tt AuthIBMDB2GroupTable} specifies the table to use to get the group info.  It defaults to the value of {\tt AuthIBMDB2UserTable}.  If you are not using groups, you do not need a {\tt groupname} field in your database, obviously.\\
\\
The optional directives {\tt AuthIBMDB2UserCondition} and \\{\tt AuthIBMDB2GroupCondition} can be used to restrict queries made against the User and Group tables. The value for each of these should be a string that you want added to the end of the where-clause when querying each table.
For example, if your user table has an {\tt active} integer field and you only want users to be able to login, if that field is 1, you could use a directive like this:\\
{\tt AuthIBMDB2UserCondition active=1}\\
\newpage
\noindent
If {\tt AuthIBMDB2UserProc} is set, the named \hyperlink{husersp}{stored procedure}\footnote{see Appendix \ref{usersp}} is responsible for returning the password of the user in question to the module. It must return exactly one value and row - the password. If set,\\ {\tt AuthIBMDB2UserTable}, {\tt AuthIBMDB2NameField}, {\tt AuthIBMDB2PasswordField}, {\tt AuthIBMDB2UserCondition} are ignored. If {\tt AuthIBMDB2NoPasswd} is {\tt On}, then the username has to be returned instead of the password. The stored procedure must have the following parameter format:
\begin{verbatim}
CREATE PROCEDURE user_procedure_name ( IN VARCHAR, OUT VARCHAR )
\end{verbatim}
If {\tt AuthIBMDB2GroupProc} is set, the named \hyperlink{hgroupsp}{stored procedure}\footnote{see Appendix \ref{groupsp}} is responsible for returning the groups the user in question belongs to. It must return an open cursor to the resultset. If set,\\ {\tt AuthIBMDB2GroupTable}, {\tt AuthIBMDB2NameField}, {\tt AuthIBMDB2GroupField}, {\tt AuthIBMDB2GroupCondition} are ignored. The stored procedure must have the following parameter format:
\begin{verbatim}
CREATE PROCEDURE group_procedure_name ( IN VARCHAR )
\end{verbatim}
If {\tt AuthIBMDB2Caching} ist set to {\tt On}, the user credentials are cached in a file defined in {\tt AuthIBMDB2CacheFile} and expires after {\tt AuthIBMDB2CacheLifetime} seconds.\\
\\
If {\tt AuthIBMDB2GroupCaching} ist set to {\tt On}, the group information is cached in a cache file that is named like the file specified in {\tt AuthIBMDB2CacheFile} but with the extension {\tt .grp}. The cache expires after {\tt AuthIBMDB2CacheLifetime} seconds.
\newpage
\subsection{Examples}
First create the two tables within DB2:
\begin{verbatim}
CREATE TABLE WEB.USERS (
    USERNAME VARCHAR(40) NOT NULL,
    PASSWORD VARCHAR(40) );

ALTER TABLE WEB.USERS
    ADD PRIMARY KEY (USERNAME);

CREATE TABLE WEB.GROUPS (
    USERNAME  VARCHAR(40) NOT NULL,
    GROUPNAME VARCHAR(40) NOT NULL );

ALTER TABLE WEB.GROUPS
    ADD PRIMARY KEY (USERNAME, GROUPNAME);
\end{verbatim}
Then you will have to insert records into the two tables:
\marginpar{\footnotesize \flushleft {\tt apr\_md5} is a User Defined Function that is explained in the db2-auth-udfs part of this documentation.}
\begin{verbatim}
INSERT INTO WEB.USERS (username, password)
    VALUES ('test', apr_md5('testpwd'));
INSERT INTO WEB.GROUPS (username, groupname) 
    VALUES ('test', 'admin');
\end{verbatim}
Then add the following lines to your httpd.conf:
\begin{verbatim}
<Directory "/var/www/my_test_dir">
    AuthName                    "DB2 Authentication"
    AuthType                    Basic
    AuthBasicProvider           ibmdb2    # Apache 2.2.x only

    AuthIBMDB2User              db2inst1
    AuthIBMDB2Password          ibmdb2
    AuthIBMDB2Database          auth
    AuthIBMDB2UserTable         web.users
    AuthIBMDB2NameField         username
    AuthIBMDB2PasswordField     passwd

    AuthIBMDB2CryptedPasswords  On
    AuthIBMDB2KeepAlive         On
    AuthIBMDB2Authoritative     On
    AuthIBMDB2NoPasswd          Off

    AuthIBMDB2GroupTable        web.groups
    AuthIBMDB2GroupField        groupname

    require                     group admin
    AllowOverride               None
</Directory>
\end{verbatim}
If you want to use \hyperlink{hsps}{stored procedures} and caching, the directives would look like this:
\begin{verbatim}
<Directory "/var/www/my_test_dir">
    AuthName                     "DB2 Authentication"
    AuthType                     Basic
    AuthBasicProvider            ibmdb2    # Apache 2.2.x only

    AuthIBMDB2User               db2inst1
    AuthIBMDB2Password           ibmdb2
    AuthIBMDB2Database           auth
    AuthIBMDB2UserProc           user_sp
    AuthIBMDB2GroupProc          group_sp

    AuthIBMDB2Caching            On
    AuthIBMDB2GroupCaching       On

    require                      group admin
    AllowOverride                None
</Directory>
\end{verbatim}
\newpage

% Include section: db2-auth-routines
\section{db2-hash-routines}
\subsection{Building the library and registering the UDFs and SPs}
Login as the instance user. Change the {\tt DB2PATH} variable in the {\tt makertn} script for your environment.
\begin{verbatim}
DB2PATH=/home/db2inst1/sqllib
\end{verbatim}
Set {\tt DB2PATH} to the directory where DB2 is accessed. This is usually the instance home directory.\\
\\
After changing the above setting, start the script\\
\\
\begin{tabular}{@{} lll @{}}
Linux and AIX & & {\tt ./makertn}\\
Win32         & & {\tt makertn.bat}\\
\end{tabular}
\\\\
The UDFs and SPs are written in ANSI C and should compile on all platforms.
You can use the {\tt bldrtn} script in your {\tt sqllib/samples/c} directory as a good start.
The only thing that you have to do is to install APR and \mbox{APR-util}.
You can get APR and APR-util at \url{http://apr.apache.org/} \\
Furthermore you need to add the compiler and linker flags for APR (see makertn).\\
\\
To register the UDFs and SPs, connect to your database and run the script:
\begin{verbatim}
db2 -tvf register.ddl
\end{verbatim}
\newpage
\subsection{Description of the UDFs and SPs}
This library delivers five routines\footnote{see Appendix \ref{rtnreference} for a reference of the UDFs and SPs}:\\
\\
\hyperlink{hphpmd5}{\tt php\_md5}\\
\hyperlink{haprmd5}{\tt apr\_md5}\\
\hyperlink{haprcrypt}{\tt apr\_crypt}\\
\hyperlink{haprsha1}{\tt apr\_sha1}\\
\hyperlink{haprsha256}{\tt apr\_sha256}\\
\hyperlink{hvalidatepw}{\tt validate\_pw}\\
\\
The {\tt php\_md5} routine is compatible to the PHP md5 function.\\
The {\tt apr\_md5}, {\tt apr\_crypt} and {\tt apr\_sha1} routines are compatible to the Apache functions that are used in the {\tt htpasswd} utility.\\
The {\tt apr\_sha256} routine returns the identifier {\tt \{SHA256\}} plus the base64 encoded sha256 hash.\\
\\
{\tt validate\_pw} can be used to validate a password against a hash.
\paragraph{Note:}{In win32 environments {\tt apr\_crypt} returns the output of {\tt apr\_md5}.}
\newpage

% Include section: scripts
\section{scripts}
\subsection{Description of the scripts}
There are four scripts to import the users and groups from already existing user and/or group files into DB2. They are written in php, so you should have the php cli binary in your {\tt /usr/local/bin} directory.\\
\\
The script {\tt sync\_pwds} is for syncing the system users with a table within your DB2 database.\\
\\
You have to change the settings in the {\tt config.php} file for your environment.\\
\\
Here is a table of the relation between the directives for the \emph{mod\_authnz\_ibmdb2} module
and the settings in the {\tt config.php} file:\\
\\
\begin{tabular}{@{} lll @{}}
{\tt config.php} & & {\tt module directive}\\
\hline
& & \\
{\tt \$dbname} & {\tt = "auth";} & {\tt AuthIBMDB2Database} \\
{\tt \$dbuser} & {\tt = "db2inst1";} & {\tt AuthIBMDB2User} \\
{\tt \$dbpwd} & {\tt = "db2inst1";} & {\tt AuthIBMDB2Password} \\
& & \\
{\tt \$usertable} & {\tt = "users";} & {\tt AuthIBMDB2UserTable} \\
{\tt \$grouptable} & {\tt = "groups";} & {\tt AuthIBMDB2GroupTable} \\
& & \\
{\tt \$namefield} & {\tt = "username";} & {\tt AuthIBMDB2NameField} \\
{\tt \$passwordfield} & {\tt = "password";} & {\tt AuthIBMDB2PasswordField} \\
{\tt \$groupfield} & {\tt = "groupname";} & {\tt AuthIBMDB2GroupField} \\
\end{tabular}
\\
\paragraph{Attention:}{The scripts were developed on Linux, therefore they will only work on systems where the {\tt /etc/passwd}, the {\tt /etc/shadow}, the {\tt /etc/group} and the {\tt /etc/gshadow} are in the same format as on Linux systems.}
\paragraph{Note:}{{\tt user\_imp} and {\tt group\_imp} will work on all systems, because these scripts don't rely on above mentioned files.}
\subsection{Examples}
If the settings in the {\tt config.php} are as above and you execute the {\tt ./user\_etc\_imp} script following happens:\\
\\
All users (except system users like root or mail) are imported from the linux box into the table {\tt users} in the database {\tt auth}.
The table {\tt users} has {\tt username} as the columnname for the users and {\tt password} as the columnname for the passwords.\\
\\
To import users from an existing htpasswd users file, just run the script
\begin{verbatim}
./user_imp <path-to-userfile>
\end{verbatim}
To import group information from an existing Apache group file, run the script
\begin{verbatim}
./group_imp <path-to-groupfile>
\end{verbatim}

\newpage

\section{CVS access}
The mod\_auth(nz)\_ibmdb2 CVS repository can be checked out through anonymous (pserver) CVS with the following instruction set. When prompted for a password for anonymous, simply press the Enter key.
\begin{verbatim}
cvs -d:pserver:anonymous@evermeet.cx:/cvs login
cvs -z3 -d:pserver:anonymous@evermeet.cx:/cvs co mod_auth_ibmdb2/module
\end{verbatim}
where {\tt module} is either {\tt mod\_authnz\_ibmdb2} or {\tt mod\_auth\_ibmdb2}.\\
\\
You can browse the CVS repository via the web at\\
\url{http://www.evermeet.cx/cvs/mod_auth_ibmdb2}
\newpage

% Include section: FAQ
\section{FAQ}
\textbf{Q:} IBM's Websphere plugin and mod\_auth(nz)\_ibmdb2 seem to break each other. What can I do? \\
\textbf{A:} mod\_auth(nz)\_ibmdb2 has to be loaded after the Websphere plugin. \\
\\
\textbf{Q:} Which versions of DB2 are supported? \\
\textbf{A:} All DB2 versions currently supported by IBM. I've tested the module with DB2 UDB v7.x, DB2 UDB v8.x, DB2 9.x, DB2 10. Older versions should work as well. \\
\\
\textbf{Q:} What is the difference between mod\_auth\_ibmdb2 and mod\_authnz\_ibmdb2? \\
\textbf{A:} mod\_authnz\_ibmdb2 is based on the new authentication backend provider scheme of Apache 2.2. This module will only work for Apache 2.2 and later. \linebreak[4] mod\_auth\_ibmdb2 works for Apache 2.0.x and 1.x. \\
\\
\textbf{Q:} What platforms are supported? \\
\textbf{A:} All POSIX platforms. I've compiled and tested the module on Linux and IBM AIX. Since the modules are using the APR libraries now, they can be compiled on Windows as well. \\
\\
\textbf{Q:} Why isn't there a binary release for .....? \\
\textbf{A:} I don't have a development environment for every operating system. Furthermore I don't think that binary releases make sense for Unix style operating systems. \\
\\
\textbf{Q:} What is the package db2-auth-routines for? \\
\textbf{A:} This package contains User Defined Functions and Stored Procedures to encrypt and validate passwords within DB2. \\
\\
\textbf{Q:} How do I get support? \\
\textbf{A:} Please submit support requests at the \href{http://sourceforge.net/tracker/?atid=633718&group_id=103064&func=browse}{Support Request Tracker} (hosted by sourceforge). \\
\\
\textbf{Q:} How do I submit bugs? \\
\textbf{A:} Please submit bugs at the \href{http://sourceforge.net/tracker/?atid=633717&group_id=103064&func=browse}{Bug Tracker} (hosted by sourceforge). \\
\\
\textbf{Q:} How do I submit a feature request? \\
\textbf{A:} Please submit feature requests at the \href{http://sourceforge.net/tracker/?atid=633720&group_id=103064&func=browse}{Feature Request Tracker} (hosted by sourceforge).
\section{Links}
\subsection{Official mod\_auth(nz)\_ibmdb2 website}
\url{http://mod-auth-ibmdb2.sourceforge.net}
\subsection{Support Requests}
\url{http://sourceforge.net/tracker/?atid=633718&group_id=103064&func=browse}
\subsection{Feature Requests}
\url{http://sourceforge.net/tracker/?atid=633720&group_id=103064&func=browse}
\subsection{Bugs}
\url{http://sourceforge.net/tracker/?atid=633717&group_id=103064&func=browse}
\subsection{Release / Update Announcement Mailing List}
\url{http://lists.sourceforge.net/lists/subscribe/mod-auth-ibmdb2-announce}
\subsection{developerWorks article}
mod\_auth\_ibmdb2: A novel authentication method for Apache\\
\url{http://www.ibm.com/developerworks/db2/library/techarticle/dm-0407tessarek/}
\newpage

% Include section: Links
\section{Links}
\subsection{Official mod\_auth(nz)\_ibmdb2 website}
\url{http://tessus.github.io/mod_authnz_ibmdb2}
\subsection{Support Requests}
\url{https://github.com/tessus/mod_authnz_ibmdb2/issues}
\subsection{PHP scripts to import users/groups}
\mbox{\url{https://github.com/tessus/usr-grp-import-scripts/archive/master.zip}}
\subsection{developerWorks article}
mod\_auth\_ibmdb2: A novel authentication method for Apache\\
\url{http://www.ibm.com/developerworks/db2/library/techarticle/dm-0407tessarek/}
\subsection{GIT repositories}
\url{https://github.com/tessus/mod_authnz_ibmdb2}\\
\url{https://github.com/tessus/mod_auth_ibmdb2}\\
\url{https://github.com/tessus/db2-hash-routines}\\
\url{https://github.com/tessus/usr-grp-import-scripts}\\

\newpage

\begin{appendix}
\section{directives and default values} \label{default}
\vspace{5ex}
\begin{tabular}{@{} lll @{}}
directive & & default value \\
\hline
& & \\
{\tt AuthIBMDB2Database} & & -- \\[0.5ex]
{\tt AuthIBMDB2Hostname} & & -- \\[0.5ex]
{\tt AuthIBMDB2Portnumber} & & {\tt 50000} \\[0.5ex]
{\tt AuthIBMDB2User} & & -- \\[0.5ex]
{\tt AuthIBMDB2Password} & & -- \\[0.5ex]
{\tt AuthIBMDB2UserTable} & & -- \\[0.5ex]
{\tt AuthIBMDB2GroupTable} & & -- \\[0.5ex]
{\tt AuthIBMDB2NameField} & & {\tt username} \\[0.5ex]
{\tt AuthIBMDB2GroupField} & & {\tt groupname} \\[0.5ex]
{\tt AuthIBMDB2PasswordField} & & {\tt password} \\[0.5ex]
{\tt AuthIBMDB2CryptedPasswords} & & {\tt yes} \\[0.5ex]
{\tt AuthIBMDB2KeepAlive} & & {\tt yes} \\[0.5ex]
{\tt AuthIBMDB2Authoritative} & & {\tt yes} \\[0.5ex] 
{\tt AuthIBMDB2NoPasswd} & & {\tt no} \\[0.5ex]
{\tt AuthIBMDB2UserCondition} & & -- \\[0.5ex]
{\tt AuthIBMDB2GroupCondition} & & -- \\[0.5ex]
{\tt AuthIBMDB2UserProc} & & -- \\[0.5ex]
{\tt AuthIBMDB2GroupProc} & & -- \\[0.5ex]
{\tt AuthIBMDB2Caching} & & {\tt off} \\[0.5ex]
{\tt AuthIBMDB2GroupCaching} & & {\tt off} \\[0.5ex]
{\tt AuthIBMDB2CacheFile} & & {\tt /tmp/auth\_cred\_cache} \\[0.5ex]
{\tt AuthIBMDB2CacheLifetime} & & {\tt 300} \\[0.5ex]
\end{tabular}
\newpage

% Include reference: UDF and SP Reference
\section{UDF and SP reference} \label{rtnreference}
\hypertarget{hbcrypt}{}
\subsection{bcrypt} \label{bcrypt1}
\begin{verbatim}
>>-BCRYPT--(--expression--)------------------------------------><

>>-BCRYPT--(--expression--,--hash--)---------------------------><
\end{verbatim}
bcrypt algorithm. The {\tt bcrypt} routine is compatible to the function used in Apache's {\tt htpasswd} utility.\\
\\
The argument can be a character string that is either a \mbox{CHAR} or \mbox{VARCHAR} not exceeding 4096 bytes.\\
\\
The result of the function is CHAR(60). The result can be null; if the argument is null, the result is the null value.\\
\\
Examples:
\begin{verbatim}
1)
    INSERT INTO USERS (username, password)
       VALUES ('test', bcrypt('testpwd'))

2)
    SELECT bcrypt( 'testpwd' ) FROM SYSIBM.SYSDUMMY1

    1
    ------------------------------------------------------------
    $2y$05$2jb66aPElSkNLT1t8e6dQepuCY2BP3JnYUh0xeV9r1PEoOGyOLkym

      1 record(s) selected.

3)
    CALL bcrypt('testpwd', ?)

      Value of output parameters
      --------------------------
      Parameter Name  : HASH
      Parameter Value : $2y$05$WYSu1X6PVA0Ra.aPSjrdv.S6hOp.AYSnNRT521rmLRjD4Mj9UY6ve

      Return Status = 0
\end{verbatim}
\newpage
\hypertarget{hsha256hex}{}
\subsection{sha256\_hex} \label{sha256hex}
\begin{verbatim}
>>-SHA256_HEX--(--expression--)--------------------------------><

>>-SHA256_HEX--(--expression--,--hash--)-----------------------><
\end{verbatim}
SHA256 algorithm. The {\tt sha256\_hex} routine returns a 64-character hexadecimal hash.\\
\\
The argument can be a character string that is either a \mbox{CHAR} or \mbox{VARCHAR} not exceeding 4096 bytes.\\
\\
The result of the function is CHAR(64). The result can be null; if the argument is null, the result is the null value.\\
\\
Examples:
\begin{verbatim}
1)
    INSERT INTO USERS (username, password)
       VALUES ('test', sha256_hex('testpwd'))

2)
    SELECT sha256_hex('testpwd') FROM SYSIBM.SYSDUMMY1

    1
    ----------------------------------------------------------------
    a85b6a20813c31a8b1b3f3618da796271c9aa293b3f809873053b21aec501087

      1 record(s) selected.

3)
    CALL sha256_hex('testpwd', ?)

      Value of output parameters
      --------------------------
      Parameter Name  : HASH
      Parameter Value : a85b6a20813c31a8b1b3f3618da796271c9aa293b3f809873053b21
    aec501087

      Return Status = 0
\end{verbatim}
\newpage
\hypertarget{hsha1hex}{}
\subsection{sha1\_hex} \label{sha1hex}
\begin{verbatim}
>>-SHA1_HEX--(--expression--)----------------------------------><

>>-SHA1_HEX--(--expression--,--hash--)-------------------------><
\end{verbatim}
SHA1 algorithm. The {\tt sha1\_hex} routine returns a 40-character hexadecimal hash.\\
\\
The argument can be a character string that is either a \mbox{CHAR} or \mbox{VARCHAR} not exceeding 4096 bytes.\\
\\
The result of the function is CHAR(40). The result can be null; if the argument is null, the result is the null value.\\
\\
Examples:
\begin{verbatim}
1)
    INSERT INTO USERS (username, password)
       VALUES ('test', sha1_hex('testpwd'))

2)
    SELECT sha1_hex('testpwd') FROM SYSIBM.SYSDUMMY1

    1
    ----------------------------------------
    98ef0758e6aac6f9a9e1197548c8190b72c9581d

      1 record(s) selected.

3)
    CALL sha1_hex('testpwd', ?)

      Value of output parameters
      --------------------------
      Parameter Name  : HASH
      Parameter Value : 98ef0758e6aac6f9a9e1197548c8190b72c9581d

      Return Status = 0
\end{verbatim}
\newpage
\hypertarget{hsha256}{}
\subsection{sha256} \label{sha256}
\begin{verbatim}
>>-SHA256--(--expression--+---------+--)-----------------------><
                          '-,--salt-'

>>-SHA256--(--expression--+---------+--,--hash--)--------------><
                          '-,--salt-'
\end{verbatim}
SHA256 algorithm. The {\tt sha256} routine returns a glibc2's crypt hash. If the system's crypt does not support sha-256, an \mbox{{\tt SQLSTATE 39702}} is returned.\\
\\
The argument can be a character string that is either a \mbox{CHAR} or \mbox{VARCHAR} not exceeding 4096 bytes.\\
An optional salt can be specified, which must be a eight-character string chosen from the set [a-z-Z0-9./]. If the salt is not exactly eight characters long, an \mbox{{\tt SQLSTATE 39703}} is returned. If the salt contains invalid characters, an \mbox{{\tt SQLSTATE 39704}} is returned.\\
\\
The result of the function is CHAR(55). The result can be null; if one of the arguments is null, the result is the null value.\\
\\
Examples:
\begin{verbatim}
1)
    INSERT INTO USERS (username, password)
       VALUES ('test', sha256('testpwd'))

2)
    SELECT sha256( 'testpwd' ) FROM SYSIBM.SYSDUMMY1

    1
    -------------------------------------------------------
    $5$S.LqPR7Z$273zPncMdmJ0dE1WdLldWVBmaHSDUDl8/tW8At8Hc0A

      1 record(s) selected.

3)
    CALL sha256('testpwd', ?)

      Value of output parameters
      --------------------------
      Parameter Name  : HASH
      Parameter Value : $5$vSDCZr2d$rfh.aDopE5l3lm26AwwcIYnuVdV7/9QBACWukqYyV3/

      Return Status = 0

4)
    SELECT sha256('testpwd', '12345678') FROM SYSIBM.SYSDUMMY1

    1
    -------------------------------------------------------
    $5$12345678$.oVAnOr/.FK8fYNiFPvoXPQvEOT9Calecygw6K9wIb9

      1 record(s) selected.

5)
    CALL sha256('testpwd', '12345678', ?)

    Value of output parameters
    --------------------------
    Parameter Name  : HASH
    Parameter Value : $5$12345678$.oVAnOr/.FK8fYNiFPvoXPQvEOT9Calecygw6K9wIb9

    Return Status = 0
\end{verbatim}
\newpage
\hypertarget{hsha512}{}
\subsection{sha512} \label{sha512}
\begin{verbatim}
>>-SHA512--(--expression--+---------+--)-----------------------><
                          '-,--salt-'

>>-SHA512--(--expression--+---------+--,--hash--)--------------><
                          '-,--salt-'
\end{verbatim}
SHA512 algorithm. The {\tt sha512} routine returns a glibc2's crypt hash. If the system's crypt does not support sha-512, an \mbox{{\tt SQLSTATE 39702}} is returned.\\
\\
The argument can be a character string that is either a \mbox{CHAR} or \mbox{VARCHAR} not exceeding 4096 bytes.\\
An optional salt can be specified, which must be a eight-character string chosen from the set [a-z-Z0-9./]. If the salt is not exactly eight characters long, an \mbox{{\tt SQLSTATE 39703}} is returned. If the salt contains invalid characters, an \mbox{{\tt SQLSTATE 39704}} is returned.\\
\\
The result of the function is CHAR(98). The result can be null; if one of the arguments is null, the result is the null value.\\
\\
Examples:
\begin{verbatim}
1)
    INSERT INTO USERS (username, password)
       VALUES ('test', sha512('testpwd'))

2)
    SELECT sha512( 'testpwd' ) FROM SYSIBM.SYSDUMMY1

    1
    ---------------------------------------------------------------------------
    -----------------------
    $6$cD33haq7$dl.RqEaLamlesTPVzSIQr4N1MY3BsVZ76VS8qNte0IOIWO2XorMg8U797KKOFGm
    X8dJhT3WuF6p17HmvvoQ6Q/

      1 record(s) selected.

3)
    CALL sha512('testpwd', ?)

      Value of output parameters
      --------------------------
      Parameter Name  : HASH
      Parameter Value : $6$1W.m9JN1$Dh.VPl7vy.igGaeDUdDWw6ZlD0xufwDWm0ukpOYknPt
    djxiSM2yzWBkzHffalb/2axNHPqEi9UUzXUbSm4LGa/

      Return Status = 0

4)
    SELECT sha512('testpwd', '12345678') FROM SYSIBM.SYSDUMMY1

    1
    ---------------------------------------------------------------------------
    -----------------------
    $6$12345678$tlHrypdWTz6FqubBpgL/ePlxr4lZuQ8OK1zfV6zWUmGJSz.5kGWwQGjg69Qm1Bm
    3.DvILruqA61o3EHsxSoko1

      1 record(s) selected.

5)
    CALL sha512('testpwd', '12345678', ?)

      Value of output parameters
      --------------------------
      Parameter Name  : HASH
      Parameter Value : $6$12345678$tlHrypdWTz6FqubBpgL/ePlxr4lZuQ8OK1zfV6zWUmGJS
    z.5kGWwQGjg69Qm1Bm3.DvILruqA61o3EHsxSoko1

    Return Status = 0
\end{verbatim}
\newpage
\hypertarget{hphpmd5}{}
\subsection{php\_md5} \label{phpmd5}
\begin{verbatim}
>>-PHP_MD5--(--expression--)-----------------------------------><

>>-PHP_MD5--(--expression--,--hash--)--------------------------><
\end{verbatim}
MD5 hash. The {\tt php\_md5} routine is compatible to the PHP md5 function.\\
\\
The argument can be a character string that is either a \mbox{CHAR} or \mbox{VARCHAR} not exceeding 4096 bytes.\\
\\
The result of the function is CHAR(32). The result can be null; if the argument is null, the result is the null value.\\
\\
Examples:
\begin{verbatim}
1)
    INSERT INTO USERS (username, password)
       VALUES ('test', php_md5('testpwd'))

2)
    SELECT php_md5( 'testpwd' ) FROM SYSIBM.SYSDUMMY1

    1
    --------------------------------
    342df5b036b2f28184536820af6d1caf

      1 record(s) selected.

3)
    CALL php_md5('testpwd', ?)

      Value of output parameters
      --------------------------
      Parameter Name  : HASH
      Parameter Value : 342df5b036b2f28184536820af6d1caf

      Return Status = 0
\end{verbatim}
\newpage
\hypertarget{haprmd5}{}
\subsection{apr\_md5} \label{aprmd5}
\begin{verbatim}
>>-APR_MD5--(--expression--)-----------------------------------><

>>-APR_MD5--(--expression--,--hash--)--------------------------><
\end{verbatim}
Seeded MD5 hash. The {\tt apr\_md5} routine is compatible to the function used in Apache's {\tt htpasswd} utility.\\
\\
The argument can be a character string that is either a \mbox{CHAR} or \mbox{VARCHAR} not exceeding 4096 bytes.\\
\\
The result of the function is CHAR(37). The result can be null; if the argument is null, the result is the null value.\\
\\
Examples:
\begin{verbatim}
1)
    INSERT INTO USERS (username, password)
       VALUES ('test', apr_md5('testpwd'))

2)
    SELECT apr_md5( 'testpwd' ) FROM SYSIBM.SYSDUMMY1

    1
    -------------------------------------
    $apr1$GfVmOTyJ$n7F1Vkwl/kX8MLgTJq1lp1

      1 record(s) selected.

3)
    CALL apr_md5('testpwd', ?)

      Value of output parameters
      --------------------------
      Parameter Name  : HASH
      Parameter Value : $apr1$GfVmOTyJ$n7F1Vkwl/kX8MLgTJq1lp1

      Return Status = 0
\end{verbatim}
\newpage
\hypertarget{haprcrypt}{}
\subsection{apr\_crypt} \label{aprcrypt}
\begin{verbatim}
>>-APR_CRYPT--(--expression--)---------------------------------><

>>-APR_CRYPT--(--expression--,--hash--)------------------------><
\end{verbatim}
Unix crypt. The {\tt apr\_crypt} routine is compatible to the function used in Apache's {\tt htpasswd} utility.\\
\\
The argument can be a character string that is either a \mbox{CHAR} or \mbox{VARCHAR} not exceeding 4096 bytes.\\
\\
The result of the function is CHAR(13). The result can be null; if the argument is null, the result is the null value.\\
\\
Examples:
\begin{verbatim}
1)
    INSERT INTO USERS (username, password)
       VALUES ('test', apr_crypt('testpwd'))

2)
    SELECT apr_crypt( 'testpwd' ) FROM SYSIBM.SYSDUMMY1

    1
    -------------
    cqs7uOvz8KBlk

      1 record(s) selected.

3)
    CALL apr_crypt('testpwd', ?)

      Value of output parameters
      --------------------------
      Parameter Name  : HASH
      Parameter Value : cqs7uOvz8KBlk

      Return Status = 0
\end{verbatim}
\newpage
\hypertarget{haprsha1}{}
\subsection{apr\_sha1} \label{aprsha1}
\begin{verbatim}
>>-APR_SHA1--(--expression--)----------------------------------><

>>-APR_SHA1--(--expression--,--hash--)-------------------------><
\end{verbatim}
SHA1 algorithm. The {\tt apr\_sha1} routine is compatible to the function used in Apache's {\tt htpasswd} utility.\\
\\
The argument can be a character string that is either a \mbox{CHAR} or \mbox{VARCHAR} not exceeding 4096 bytes.\\
\\
The result of the function is CHAR(33). The result can be null; if the argument is null, the result is the null value.\\
\\
Examples:
\begin{verbatim}
1)
    INSERT INTO USERS (username, password)
       VALUES ('test', apr_sha1('testpwd'))

2)
    SELECT apr_sha1( 'testpwd' ) FROM SYSIBM.SYSDUMMY1

    1
    ---------------------------------
    {SHA}mO8HWOaqxvmp4Rl1SMgZC3LJWB0=

      1 record(s) selected.

3)
    CALL apr_sha1('testpwd', ?)

      Value of output parameters
      --------------------------
      Parameter Name  : HASH
      Parameter Value : {SHA}mO8HWOaqxvmp4Rl1SMgZC3LJWB0=

      Return Status = 0
\end{verbatim}
\newpage
\hypertarget{haprsha256}{}
\subsection{apr\_sha256} \label{aprsha256}
\begin{verbatim}
>>-APR_SHA256--(--expression--)--------------------------------><

>>-APR_SHA256--(--expression--,--hash--)-----------------------><
\end{verbatim}
SHA256 algorithm. The {\tt apr\_sha256} routine returns the identifier {\tt \{SHA256\}} plus the base64 encoded sha256 hash.\\
\\
The argument can be a character string that is either a \mbox{CHAR} or \mbox{VARCHAR} not exceeding 4096 bytes.\\
\\
The result of the function is CHAR(52). The result can be null; if the argument is null, the result is the null value.\\
\\
Examples:
\begin{verbatim}
1)
    INSERT INTO USERS (username, password)
       VALUES ('test', apr_sha256('testpwd'))

2)
    SELECT apr_sha256( 'testpwd' ) FROM SYSIBM.SYSDUMMY1

    1
    ----------------------------------------------------
    {SHA256}qFtqIIE8Maixs/NhjaeWJxyaopOz+AmHMFOyGuxQEIc=

      1 record(s) selected.

3)
    CALL apr_sha256('testpwd', ?)

      Value of output parameters
      --------------------------
      Parameter Name  : HASH
      Parameter Value : {SHA256}qFtqIIE8Maixs/NhjaeWJxyaopOz+AmHMFOyGuxQEIc=

      Return Status = 0
\end{verbatim}
\newpage
\hypertarget{hvalidatepw}{}
\subsection{validate\_pw} \label{validatepw}
\begin{verbatim}
>>-VALIDATE_PW--(--password--,--hash--)------------------------><

>>-VALIDATE_PW--(--password--,--hash--,--is_valid--)-----------><
\end{verbatim}
This routine can be used to validate a password against a hash.\\
\\
The two input arguments can be character strings that are either a CHAR or VARCHAR not exceeding 4096 bytes (password) and 120 bytes (hash). The second parameter (hash) must not be empty, otherwise an \mbox{{\tt SQLSTATE 39701}} is returned.\\
\\
The result of the routine is an INTEGER. If the password is valid, 1 is returned. If the password is not valid, 0 is returned. The result can be null; if the argument is null, the result is the null value.\\
\\
Examples:
\begin{verbatim}
1)
    SELECT validate_pw('testpwd', 'cqs7uOvz8KBlk') FROM SYSIBM.SYSDUMMY1"

    1
    -----------
              1

      1 record(s) selected.

2)
    CALL validate_pw('testpwd', 'cqs7uOvz8KBlk', ?)

      Value of output parameters
      --------------------------
      Parameter Name  : IS_VALID
      Parameter Value : 1

      Return Status = 0

3)
    CALL validate_pw('testpwd', '0123456789abcdef', ?)

      Value of output parameters
      --------------------------
      Parameter Name  : IS_VALID
      Parameter Value : 0

      Return Status = 0
\end{verbatim}

\newpage

\hypertarget{hsps}{}
\section{Stored Procedure Support} \label{sp}
Stored procedures can minimize the network traffic and with regard to the authentication module configuration they can ease the administration. The module supports two types of stored procedures: one for user authentication and one for group authentication.\\
\\
For the following 2 sections we use these 3 tables:
\\
\begin{verbatim}
CREATE TABLE WEB.USERS (
    USERNAME VARCHAR(40) NOT NULL,
    PASSWORD VARCHAR(40) );

ALTER TABLE WEB.USERS
    ADD PRIMARY KEY (USERNAME);

CREATE TABLE WEB.GROUPS (
    GROUPNAME VARCHAR(40) NOT NULL,
    ACTIVE    INTEGER     NOT NULL );

ALTER TABLE WEB.GROUPS
    ADD PRIMARY KEY (GROUPNAME);

CREATE TABLE WEB.MAPPING (
    USERNAME  VARCHAR(40) NOT NULL,
    GROUPNAME VARCHAR(40) NOT NULL );

ALTER TABLE WEB.MAPPING
    ADD PRIMARY KEY (USERNAME, GROUPNAME)
    ADD FOREIGN KEY (USERNAME) REFERENCES WEB.USERS (USERNAME)
    ADD FOREIGN KEY (GROUPNAME) REFERENCES WEB.GROUPS (GROUPNAME);
\end{verbatim}
\newpage
\hypertarget{husersp}{}
\subsection{user authentication} \label{usersp}
The stored procedure for user authentication is responsible for returning the password of the user in question to the module. It must return exact one value - the password. If {\tt AuthIBMDB2NoPasswd} is {\tt On}, then the username has to be returned instead of the password.\\
\\
The stored procedure must have the following parameter format:
\begin{verbatim}
CREATE PROCEDURE user_procedure_name ( IN VARCHAR, OUT VARCHAR )
\end{verbatim}
Example:
\begin{verbatim}
CREATE PROCEDURE user_sp
(IN v_username VARCHAR(40), OUT v_password VARCHAR(40))
LANGUAGE SQL
BEGIN
  SELECT password INTO v_password FROM web.users 
  WHERE username = v_username;
END@
\end{verbatim}
If {\tt AuthIBMDB2NoPasswd} is {\tt On}, then the stored procedure would have to look like this:
\begin{verbatim}
CREATE PROCEDURE user_sp
(IN v_username VARCHAR(40), OUT v_password VARCHAR(40))
LANGUAGE SQL
BEGIN
  SELECT username INTO v_password FROM web.users 
  WHERE username = v_username;
END@
\end{verbatim}
\newpage
\hypertarget{hgroupsp}{}
\subsection{group authentication} \label{groupsp}
The stored procedure for group authentication is responsible for returning the groups the user in question belongs to. It must return an open cursor to the resultset.\\
\\
The stored procedure must have the following parameter format:
\begin{verbatim}
CREATE PROCEDURE group_procedure_name ( IN VARCHAR )
\end{verbatim}
Example
\begin{verbatim}
CREATE PROCEDURE group_sp
(IN v_username VARCHAR(40))
LANGUAGE SQL
DYNAMIC RESULT SETS 1
BEGIN
  DECLARE res CURSOR WITH RETURN FOR
  SELECT m.groupname FROM web.groups g, web.mapping m 
  WHERE m.groupname = g.groupname AND
        m.username = v_username AND
        g.active = 1;

  OPEN res;
END@
\end{verbatim}

\end{appendix}

\begin{center}
\vspace{\fill}\ \newline
{\tiny \rm $ $RCSfile$ $ }
{\tiny \rm $ $Date$ $ }
{\tiny \rm $ $Revision$ $ }
\end{center}

\end{document}
